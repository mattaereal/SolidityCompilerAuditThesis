\thispagestyle{plain}
\begin{center}
    \Large
    \textbf{Auditoría de Software Orientada a Compiladores}

    \vspace{0.2cm}
    \large
    {Caso de Estudio: Solidity}
 
    \vspace{0.4cm}
    \textbf{{Matías Ariel Ré Medina}\\
    {Dr. Ing. José María Massa}
    }
 
    \vspace{0.9cm}
    \textbf{Abstract}
\end{center}


Con el advenimiento de nuevas tecnologías y la necesidad constante de seguir desarrollando software debido a las demandas del mercado, es inevitable depender cada vez de más herramientas externas para mantenerse al día. Pero realmente quienes desarrollan, ¿entienden la gravedad que posee cada vez depender más, ciegamente, de otras tecnologías para crear nuevas? Con tanto acoplamiento, sólo basta que un eslabón de la cadena sea inseguro para que todo el desarrollo también lo sea. Debido a que el compilador es el unico software que tiene la posibilidad de mirar (casi) todas las lıneas de un software, el enfoque que propone esta tésis parte de una observación a la responsabilidad que se deposita del lado del lenguaje en el que programan desarrolladores, sin preguntarse si lo que están compilando introduce posibles problemáticas. El documento de tesis comprende una puesta al día de las técnicas disponibles para realizar auditorías de sistemas de software en general y en particular de aquellas utilizables en el análisis de compiladores. Asimismo, se presenta el trabajo de auditoría sobre el lenguaje de programación Solidity y su compilador solc. Éste comprende en detalle tanto los procesos como las herramientas utilizadas para la auditoría. El lenguaje Solidity se encuentra dentro de aquellos lenguajes orientados al manejo de Smart Contracts y su importancia es crítica debido a que deben poseer una ejecución verificable y observable. Algunas de las aplicaciones de los Smart Contracts son en el campo de las finanzas, los seguros  y contratos en general. Se presentan además algunas soluciones y tecnologías existentes que pueden ser aplicadas a la auditoría de compiladores, luego se propone una metodología específica para la auditoría objeto de esta tesis y finalmente se presentan los resultados obtenidos desde el punto de vista del cliente interesado en esta auditoría, junto con las conclusiones, y posibles extensiones de este trabajo.