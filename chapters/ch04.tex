

\section{Introducción}

El objetivo de la metodología a seguir consiste en revisar el compilador y lenguaje de Solidity, analizar su diseño general y arquitectura, y reportar potenciales vulnerabilidades de seguridad que puedan llegar a comprometer el código compilado. En este capítulo se describe el trabajo realizado presentando ejemplos de observaciones en áreas específicas del código que presentan problemas concretos, así como también observaciones generales que atraviesan el proyecto entero, que puede mejorar su calidad como un todo.\\

Se puede interpretar a la metodología como una auditoría abarcativa, no sólo de seguridad, sino en todos los aspectos que permitan encontrar problemas. No es menor resaltar que cualquier problema en este contexto puede ser considerado un potencial impacto de seguridad.\\

El método propuesto no tiene intención de destacarse por utilizar características novedosas a nivel tecnológico o introducir nuevos procesos. De hecho todo lo que se realizará en esta sección, no es más que, en distintas medidas, una conjunción de los dos capítulos anteriores, y es por eso que parece importante hacer énfasis en este tipo de propuestas, ya que están al alcance del estado del arte.\\

La razón de centrar el foco de investigación en un proyecto open source, que posee instrucciones sobre cómo construirlo, y una documentación extensa, es para presentar una situación favorable y menos limitante para realizar una auditoría. No obstante, poseer esta situación lo hace más complejo y desafiante, ya que se puede analizar desde todas las perspectivas posibles.\\

Su código está disponible online, y se ha decidido congelar el repositorio en la última versión estable a la hora de realizar esta investigación.

\section{Alcance}
